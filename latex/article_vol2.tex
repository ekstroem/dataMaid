\documentclass[article,shortnames]{jss}
\usepackage[utf8]{inputenc}
\usepackage[nogin]{Sweave}
\usepackage{natbib}
\usepackage{pdfpages}
\usepackage{xspace}
\usepackage{array}
\usepackage{tikz}
\usetikzlibrary{shapes.geometric, arrows}

\tikzstyle{io} = [trapezium, trapezium left angle=70, trapezium right angle=110, minimum width=3cm, minimum height=1cm, text centered, draw=black, fill=blue!30]
\tikzstyle{process} = [rectangle, minimum width=3cm, minimum height=1cm, text centered, draw=black, fill=orange!30]
\tikzstyle{decision} = [diamond, minimum width=3cm, minimum height=1cm, text centered, draw=black, fill=green!30]
\tikzstyle{arrow} = [thick,->,>=stealth]

\newcommand{\hl}[1]{\textcolor{magenta}{#1}}

\renewcommand\topfraction{.9}
\renewcommand\textfraction{.1}
\renewcommand{\floatpagefraction}{.6}

\renewcommand{\topfraction}{0.85}
\renewcommand{\bottomfraction}{0.85}
\renewcommand{\textfraction}{0.15}
\renewcommand{\floatpagefraction}{0.7}

%Virker ikke:
%\newcommand{\R}{\proglang{R}\xspace}

\newcommand{\R}[1]{\code{#1}}

\newcolumntype{L}[1]{>{\raggedright\let\newline\\\arraybackslash\hspace{0pt}}p{#1}}
\newcolumntype{C}[1]{>{\centering\let\newline\\\arraybackslash\hspace{0pt}}p{#1}}
\newcolumntype{R}[1]{>{\raggedleft\let\newline\\\arraybackslash\hspace{0pt}}p{#1}}


%%%%%%%%%%%%%%%%%%%%%%%%%%%%%%
%% declarations for jss.cls %%%%%%%%%%%%%%%%%%%%%%%%%%%%%%%%%%%%%%%%%%
%%%%%%%%%%%%%%%%%%%%%%%%%%%%%%

%% almost as usual
\author{Anne H. Petersen\\Biostatistics\\Department of Public
  Health\\University of Copenhagen \And Claus Thorn Ekstr\o m\\Biostatistics\\Department of Public
  Health\\University of Copenhagen}
\title{\pkg{cleanR}: Maid for Cleaning Datasets in \proglang{R}}

%% for pretty printing and a nice hypersummary also set:
\Plainauthor{Anne H. Petersen, Claus Thorn Ekstr\o m} %% comma-separated
\Plaintitle{{cleanR}: maid for cleaning datasets in R} %% without formatting
\Shorttitle{\pkg{cleanR}: maid for cleaning datasets in R} %% a short title (if necessary)

%% an abstract and keywords
\Abstract{Data cleaning and -validation are the first steps in any
  data analysis, as the validity of the conclusions from the analysis
  hinges on the quality of the input data. Mistakes in the data can
  arise for any number of reasons, including erroneous codings,
  malfunctioning measurement equipment, inconsistent data generation
  manuals and many more.  Ideally, a human investigator should go
  through each variable in the dataset and look for potential errors
  --- both in input values and codings --- but that process can be very
  time-consuming, expensive and error-prone in itself.

  We describe an \proglang{R} package which implements an extensive
  and customizeable suite of quality assessment tools that can be applied to
  a dataset in order to identify potential problems in its
  variables. The results can be presented in an auto-generated,
  non-technical, stand-alone overview document, intended to be perused
  by an investigator with an understanding of the variables in the
  data, but not necessarily knowledge of \R{R}. Thereby, \pkg{cleanR} 
  aids the dialogue between data
  analysts and field experts, while also providing easy
  documentation of reproducible data cleaning steps and data quality
  control. Moreover, the \pkg{cleanR} solution changes the data
  cleaning process from the usual ad hoc approach to a systematic,
  well-documented endeavor.  \pkg{cleanR} also provides a suite of
  more typical \proglang{R} tools for interactive data quality
  assessment and -cleaning, where the data inspections all live and die
  within the \proglang{R} console. 
    
  % The \pkg{cleanR} package is designed to be easily extended with
  % custom user-created checks that are relevant in particular
  % situations. \hl{Already said that above. Either expand on it or
  % delete this.}
}
\Keywords{data cleaning, quality control, \proglang{R}}
\Plainkeywords{data cleaning, quality control, R} %% without formatting
%% at least one keyword must be supplied

%% publication information
%% NOTE: Typically, this can be left commented and will be filled out by the technical editor
%% \Volume{50}
%% \Issue{9}
%% \Month{June}
%% \Year{2012}
%% \Submitdate{2012-06-04}
%% \Acceptdate{2012-06-04}

%% The address of (at least) one author should be given
%% in the following format:
\Address{
  Claus Thorn Ekstr\o m\\
  Biostatistics, Department of Public Health\\
  University of Copenhagen\\
  Denmark\\
  E-mail: \email{ekstrom@sund.ku.dk}\\
  URL: \url{http://staff.pubhealth.ku.dk/~ekstrom/}
}
%% It is also possible to add a telephone and fax number
%% before the e-mail in the following format:
%% Telephone: +43/512/507-7103
%% Fax: +43/512/507-2851

%% for those who use Sweave please include the following line (with % symbols):
%% need no \usepackage{Sweave.sty}

%% end of declarations %%%%%%%%%%%%%%%%%%%%%%%%%%%%%%%%%%%%%%%%%%%%%%%


\begin{document}

\section{Introduction}
Though data cleaning might be regarded as a somewhat tedious activity,
adequate data cleaning is crucial in any data analysis. With
ever-growing dataset sizes and complexities, statisticians and data
analysts find themselves spending a large portion of their time on
data cleaning and on data wrangling. While a computer should never
make unsupervised decisions on what should be done to potential
errors in the dataset, it can still be an extremely useful tool in the 
data cleaning process. Some errors can be tracked down and flagged by a 
computer without further ado, while other types of errors need an analytic
context in order to shine through. But even in the latter case, well-designed
software can aid the process tremendously by giving the human investigator the 
information needed for identifying issues. 

Online tools such as OpenRefine (\url{http://openrefine.org/}) and
\proglang{R}-packages such as \pkg{plyr}, and \pkg{data.table} have
made data wrangling a lot easier, but only a handful of packages such
as \pkg{editrules}, \pkg{validate}, \pkg{DataCombine}, and
\pkg{janitor} attempt to implement systematic, reproducible data
cleaning.  These packages use different approaches for data cleaning:
\pkg{editrules} and \pkg{validate} provide frameworks for setting up
and checking constraints on the variables, while \pkg{DataCombine} and
\pkg{janitor} both provide a few functions for identifying problems
(e.g, duplicates, dates coded as numbers, etc.) in data.

While these tools attempt to alleviate the ubiquitous ad hoc approach
to data cleaning, they are primarily intended for the data savvy users
and less so for the general researcher with a knowledge about a specific
field and the context of the available data. The \pkg{cleanR} package
tries to address this by providing a framework that both allows for
extendable, systematic, reproducible data cleaning, and summarizing
findings for researchers from other fields such that they can act as
human experts when tracking down potential errors. 

% \begin{itemize}
% \item Makes the point that data cleaning is usually done in a very ad hoc manner
% \item Maybe hints to the discussion later about poor documentation of data cleaning/quality assessment as the status quo for most analysts
% \item Gives a specification of what we mean by data errors/mistakes/whatever. I'm not completely sure this will be clear to everyone and we should probably mention it early on to not lose readers/users.
% \end{itemize}
% }

Data cleaning is a time consuming endeavor, as it inherently requires
human interaction since every dataset is different and the variables
in the dataset can only be understood in the proper context of their
origin. This often requires a collaborative effort between an expert
in the field and a statistician or data scientist, which may be why
the process of proper data cleaning is not always undertaken. In many
situations, these errors are discovered in the process of the data
analysis (e.g., a categorical variable with numeric labels for each
category may be wrongly classified as a quantitative variable or a
variable where all values have erroneously been coded to the same
value), but in other cases a human with knowledge about the data
context area is needed to identify possible mistakes in the data
(e.g., if there are 4 categories for a variable that should only have
3).

The \pkg{cleanR} approach to data cleaning and -quality assessment is
governed by two fundamental paradigms. First of all, there is no need
for data cleaning to be an ad hoc procedure. Often, we have a very
clear idea of what flags are raisable in a given dataset before we
look at it, as we were the ones to produce it in the first place. This
means that data cleaning can easily be a well-documented,
well-specified procedure. In order to aid this paradigm, \pkg{cleanR}
provides easy-to-use, automated tools for data quality assessment in
\proglang{R} on which data cleaning decisions can be made. This quality
assessment is presented in an auto-generated overview document,
readable by data analysts and field experts alike, thereby also
contributing to an inter-field dialogue about the data at
hand. Oftentimes, e.g. distinguishing between faulty codings of a
numeric value and unusual, but correct, values requires
problem-specific expertise that might not be held by the data
analyst. Hopefully, having easy access to data descriptions through
\pkg{cleanR} will help this necessary knowledge sharing.

While \pkg{cleanR}s primary raison d'être is auto-generating data
quality assessment overview documents, we still wish to emphasize that
it is \emph{not} a tool for unsupervised data cleaning. This qualifies
as the second paradigm of \pkg{cleanR}: Data cleaning decisions
should always be made by humans. Therefore, \pkg{cleanR} does not
supply any tools for ``fixing'' errors in the data. However, we do
provide interactive functions that can be used to identify potentially
erroneous entries in a dataset and that can make it easier to solve
data issues, one variable at a time.


This manuscript is structured as follows: First, in Section
\ref{sec:usingcleanR}, we introduce the main representative of the first
paradigm, namely the \R{clean()} function, which generates data
cleaning overview documents. In the \pkg{cleanR} package, we have
provided a number of default cleaning steps that cover the data
cleaning challenges, we find to be most common. Next, in Section
\ref{sec:interactiveCleanR}, we present the interactive mode of \pkg{cleanR}, as motivated
by the second paradigm above. But, as any data analyst knows,
every dataset is different, and some datasets might include problems
that cannot be detected by our data checking functions. In Section
\ref{sec:extending}, we turn to the question of how \pkg{cleanR} extensions
can be made, such that they are integrable with the \R{clean()}
function and with the other tools available in \pkg{cleanR}.  At last,
in Section \ref{sec:specificExamples}, we discuss a number of examples of
specific data cleaning challenges and how \pkg{cleanR} can be used to
solve them.

% \hl{Is there a Section 6?}

%\hl{Do they typically discuss notation in articles like this? I try (somewhat inconsistently) to refer to functions as \R{function()} while other \R{R} objects are referred to merely as \R{object}. This is also the style adopted by Wickham in his books.}
 

%% include your article here, just as usual
%% Note that you should use the \pkg{}, \proglang{} and \code{}
%% commands.


\section{Creating a data cleaning overview} 
\label{sec:usingcleanR}

The \R{clean()} function is the primary workhorse of \pkg{cleanR} and
this is the only function needed if a standard battery
of tests are used to generate data cleaning summaries. The data
cleaning summary itself is an overview document, intended for reading
by humans, in either pdf or html format. Appendix \ref{sec:appendix1}
provides an example of a data cleaning output document, produced by
calling \R{clean()} on the dataset \R{toyData} available in
\pkg{cleanR}. The first two pages of this data cleaning output are
shown in Figure~\ref{fig:example1}. \R{toyData} is a very
small ($15$ rows of $6$ variables), artificial dataset which was created to
illustrate the main capabilities of \pkg{cleanR}. The following
commands load the dataset and produce the cleaning output:

\begin{figure}[tb]
\begin{center}
\frame{\includegraphics[width=7.5cm,page=2]{cleanR_toyData.pdf}} 
\frame{\includegraphics[width=7.5cm,page=3]{cleanR_toyData.pdf}}
%\includepdf[pages={2}, pagecommand={}]{cleanR_testData.pdf}
\end{center}
\label{fig:example1}
\caption{Example output from running \R{clean()} on the \R{toyData}
  dataset. First a summary of the full dataset is given and then
  type-dependent information on each variable is given in a table and
  a graph. Larger versions of the pages can be seen in
  Appendix~\ref{sec:appendix1}.}
\end{figure}



\begin{Schunk}
\begin{Sinput}
> library(cleanR)
> data(toyData)
> toydata
\end{Sinput}
\begin{Soutput}
   var1 var2  var3        var4 var5       var6
1   red    1     a -0.65959383    1 Irrelevant
2   red    1     a  0.08671649    2 Irrelevant
3   red    1     a -0.10951326    3 Irrelevant
4   red    2     a  0.08630221    4 Irrelevant
5   red    2     a -1.84311184    5 Irrelevant
6   red    6     b  0.92210680    6 Irrelevant
7   red    6     b  1.01921086    7 Irrelevant
8   red    6     b -0.92428326    8 Irrelevant
9   red  999     c -0.65340163    9 Irrelevant
10  red   NA     c  0.21133941   10 Irrelevant
11 blue    4     c  0.91783009   11 Irrelevant
12 blue   82     .  0.10313983   12 Irrelevant
13 blue   NA        0.16954218   13 Irrelevant
14 <NA>  NaN other  0.41967230   14 Irrelevant
15 <NA>    5 OTHER  0.77143836   15 Irrelevant
\end{Soutput}
\begin{Sinput}
> clean(toyData)
\end{Sinput}
\end{Schunk}

By default, an \proglang{R} markdown file and a rendered pdf overview
document is produced, saved to the disc (in the working directory) and
opened for immediate inspection. Turning to Figure~\ref{fig:example1},
we see that such a data cleaning output document consists of two
parts. First, an overview of what was done is presented under the
title \textit{Data cleaning summary}. Secondly, each variable in the
dataset is presented in turn using (up to) three tools in the
\textit{Variable list}: A table summarizing key features of the
variable, a figure visualizing its distribution and a
list of flagged issues, if any. For instance, in the \R{numeric}-type variable
\R{var2} from \R{toyData}, \R{clean()} has identified two values that
are suspected to be miscoded missing values (\R{999} and \R{NaN}),
while two values were also flagged as potential outliers that should
be investigated more carefully.
%We can then return to Part
%1, the data cleaning summary, and inspect what sorts of checks were
%performed on this variable. \hl{hm, delete the last sentence? or use
%  different variable? Only these exact two checks were performed}. 

Though the \R{clean()} function is very easy to use, it should not be
mistaken to be inflexible. A large number of function arguments allows
for the cleaning overview document to be molded according to the
user's needs. The most commonly used arguments are summarized in
Table~\ref{table.cleanFormals} and they are grouped according to the
part of the cleaning process they influence. In order to understand
this distinction, a glimpse of the inner structure of \R{clean()} is
shown in Figure~\ref{figure:cleanStructure}. Below, we present a few
examples on how to use the arguments from Table \ref{table.cleanFormals}
 to influence the output of a \R{clean()} call.

\begin{table}
\small
\begin{tabular}{p{0.25\linewidth}p{0.45\linewidth}p{0.2\linewidth}}
\hline
Argument & Description & Default value \\
\hline

\smallskip Control input variables and summary\\
\quad \R{useVar} & What variables should be used? & \R{NULL} (corresponding to all variables) \\
\quad \R{ordering} & Ordering of the variables in the data summary (as is or alphabetical) & \R{"asIs"} \\
\quad \R{onlyProblematic} & Should only variables flagged as problematic be included in the \textit{Variable list}? & \R{FALSE} \\
\quad \R{listChecks} & Should an overview of what checks were performed be listed in the \textit{Data cleaning summary}? &  \R{TRUE} \\
\quad \R{preChecks} & What check functions should be called to determine whether a variable is suitable for summarization, visualization and checking? & \R{c("isKey", "isEmpty")}

\smallskip Control summarize, visualize, and check steps \\
\quad \R{mode} & What steps should be performed for each variable (out
                 of the three possibilities \textit{summarize},
                 \textit{visualize}, \textit{check})? &
                                                        \R{c("summarize", "visualize", "check")} (corresponding to all three steps) \\
\quad \R{smartNum} & Should numerical values with only a few unique
                     levels be flagged and treated as a factor variable? & \R{TRUE} \\
\quad \R{maxProbVals} & Maximum number of problematic values to print, if any are found in data checks & \R{Inf} \\
\quad \R{maxDecimals} & Maximum number of decimals to print for numeric values in the variable list & \R{2} \\
\quad \R{twoCol} & Should the summary table and visualizations be placed side-by-side (in two columns)? & \R{TRUE} \\

\smallskip Control output and post-processing \\
\quad \R{output} & Type of output file to be produced (html, or pdf) & \R{"pdf"} \\
\quad \R{render} & Should the output file be rendered from markdown? & \R{TRUE} \\
\quad \R{openResult} & If a  pdf/html file is rendered, should it
                       automatically open afterwards, and if not,
                       should the \R{rmarkdown} file be opened? & \R{TRUE} \\
\quad \R{replace} & Overwrite an existing files with the same name? & \R{FALSE} \\

\hline
\end{tabular}
\caption{A selection of commonly used arguments to \R{clean()} separated into the parts they control.}
\label{table.cleanFormals}
\end{table}


% Below, we present a selection of these arguments in
% sections \hl{xx} and \hl{xxx}. We do this under two distinct headlines
% to emphasize that two levels of control are available when using
% \R{clean()}. Either, we can control the \hl{[something] - essentially,
%   this is everything but SVC-functions - } or we can control what
% quality assessments and descriptions each variable are subjected
% to.  \hl{segway to next bit or
%   just something to end the paragraph}

\begin{figure}[tb]
% \begin{description}
% \item[Input] A dataset
% 	\begin{itemize} 
% 		\item This should be of type \R{data.frame} or \R{tibble} \hl{or data.table?}
% 	\end{itemize}
% \item[Create contents] For each variable in the dataset, do the following:
% 	\begin{description}
% 		\item[Stage 1:] Pre-checks 
% 			\begin{itemize}
% 				\item Is the variable suitable for summarization, visualization and checks?
% 				\begin{description}
% 					\item[Yes:] Go to stage 2. 
% 					\item[No:] Move on to the next variable.
% 				\end{description}
% 			\end{itemize}
% 		\item[Stage 2:] SVC-steps
% 			\begin{description}
% 				\item[Summarize] Call \R{summarize()} to produce a summary table describing the variable. What features enter this table depends on the data class of the variable.
% 				\item[Visualize]  Call \R{visualize()} to produce a plot visualizing the distribution of the variable. 
% 				\item[Check]  Call \R{check()} to apply quality- and error checks to the variable. What checks are used depends on the data class of the variable.
% 			\end{description}
% 	\end{description}
% \item[Output] Files for a overview document are saved to the disc and possibly also opened.
% 	\begin{itemize}
% 		\item Always a \R{rmarkdown} (.Rmd) file
% 		\item Possible also a html or pdf file
% 	\end{itemize}
% \end{description}

% Define block styles
\tikzstyle{decision} = [diamond, draw, fill=blue!20, 
    text width=5.5em, text badly centered, node distance=3cm, inner
    sep=0pt]
\tikzstyle{block} = [rectangle, draw, fill=blue!20, 
    text width=6em, text centered, rounded corners, minimum
    height=4em]
\tikzstyle{line} = [draw, -latex']
\tikzstyle{cloud} = [draw, ellipse,fill=red!20, node distance=3cm,
    text width=5em,
    minimum height=2em]

\begin{center}    
\begin{tikzpicture}[node distance = 2cm, auto,thick,scale=0.75, every node/.style={transform shape}]
    % Place nodes
    \node [block] (init) {Get next variable and run pre-checks};
    \node [cloud, above of=init] (input) {Input \texttt{data.frame} or \texttt{tibble}};
%    \node [cloud, right of=init] (system) {system};
    \node [decision, right of=init, node distance=4cm] (precheck) {Is variable suitable for inclusion};
    \node [block, right of=precheck, node distance=4cm] (summarize)
    {Run \texttt{summarize()} to produce summary table};
    \node [block, below of=summarize, node distance=3cm] (visualize)
    {Run \texttt{visualize()} to plot variable};
    \node [block, below of=visualize, node distance=2.5cm] (check)
    {Call \texttt{check()} to run error checks};
    \node [decision, below of=check, node distance=2.7cm] (done) {More
      variables?};
    \node [block, right of=done, node distance=3.5cm] (stop) {Write
      \proglang{R} markdown file};
    \node [cloud, below of=stop, node distance=3.5cm] (render) {Render
      markdown and possiby open};
    % Draw edges
    \path [line] (summarize) -- (visualize);
    \path [line] (visualize) -- (check);
    \path [line] (check) -- (done);
    \path [line] (done.south) -- +(0,-10pt) -| node [near start] {yes} (init);
%    \path [line] (identify) -- (evaluate);
%    \path [line] (evaluate) -- (decide);
%    \path [line] (init) -| node [near start] {yes} (precheck);
    \path [line] (init) -- (precheck);
    \path [line] (precheck) -- node [near start] {yes} (summarize);
    \path [line] (precheck) |- node [near start] {no} (done);
    \path [line] (done) -- node [near start] {no} (stop);
%    \path [line] (update) |- (identify);
%    \path [line] (decide) -- node {no}(stop);
    \path [line,dashed] (input) -- (init);
    \path [line] (stop) -- (render);
%    \path [line,dashed] (system) -- (init);
%    \path [line,dashed] (system) |- (evaluate);
\end{tikzpicture}
\end{center}

\caption{Schematic illustration of the stages undertaken when running
  \R{clean()}. Each variable is checked for eligibility before
  running \R{summarize()}, \R{visualize()}, and \R{check()}, and the
  resulting \proglang{R} markdown file may be rendered and opened.}
\label{figure:cleanStructure}
\end{figure}




% \subsection{Controlling [something]}
% \label{subsection:controlSomething}

\subsection{Polishing off the arguments}
We begin with an example that is intended as an illustration of how \R{clean()} might
be used in the very first stages of data cleaning, where we are still uncertain
about exactly what is needed and how much times should be allocated to data
cleaning. At this stage, what is really needed, if a very rough idea of the extends of
errors in the dataset. In this scenario, we might wish to obtain a summary document in html format 
that only contains the variables with potential problems, and with a limit of
maximum 10 printed potential errors for each variable we can write
(output not shown). Also, we can add the argument \R{replace=TRUE} in
order to force \R{clean()} to overwrite any existing files produced by
\R{clean()}, as we might run this function a few times in a row before deciding
exactly which cleaning overview we want. Using the \R{toyData} dataset as a guinea pig, we type:

\begin{Schunk}
\begin{Sinput}
> clean(toyData, output = "html", onlyProblematic = TRUE, maxProbVals = 10,
        replace = TRUE)
\end{Sinput}
\end{Schunk}

The final rendering of the generated markdown file is controlled by
the \R{render} and \R{openResult} arguments, which both default to
\R{TRUE}. \R{render} determines if the \proglang{R} markdown file
produced should be rendered using the \pkg{rmarkdown} package and
\R{openResult} decides whether the output html or pdf file should be
opened. The following command produces an \proglang{R} markdown file
containing the data cleaning summary information, but without
rendering nor opening the markdown file:

\begin{Schunk}
\begin{Sinput}
> clean(toyData, output="html", render=FALSE, openResult=FALSE)
\end{Sinput}
\end{Schunk}


\begin{table}
\centering
\begin{tabular}{p{0.35\linewidth} p{0.3\linewidth} p{0.01\linewidth} p{0.01\linewidth} p{0.01\linewidth} p{0.01\linewidth} p{0.01\linewidth}
 p{0.01\linewidth} p{0.01\linewidth}}
  \hline
& Description &  \multicolumn{7}{c}{Variable classes} \\ \smallskip
 & &  C & F & I & L & B & N & D\\ 
  \hline \smallskip
  \textbf{\R{summaryFunction}s}  \smallskip \\
  \quad \R{centralValue} & Compute median or mode &  $\times$ & $\times$ & $\times$ & $\times$ & $\times$ & $\times$ & $\times$ \\ 
  \quad \R{countMissing} & Compute ratio of missing observations &  $\times$ & $\times$ & $\times$ & $\times$ & $\times$ & $\times$ & $\times$  \\ 
  \quad \R{minMax} & Find minimum and maximum values &   &  & $\times$ & &  & $\times$ & $\times$  \\ 
  \quad \R{quartiles} & Compute 1st and 3rd quartiles &    &  & $\times$ & &  & $\times$ & $\times$ \\ 
  \quad \R{uniqueValue} & Count number of unique values &   $\times$ & $\times$ & $\times$ & $\times$ & $\times$ & $\times$ & $\times$  \\ 
  \quad \R{variableType} & Data class of variable & $\times$ & $\times$ & $\times$ & $\times$ & $\times$ & $\times$ & $\times$  \\ 
  \smallskip \\
 \textbf{\R{visualFunction}s} \smallskip \\
  \quad \R{basicVisual} & Histograms and barplots using base \proglang{R} graphics &  $\times$ & $\times$ & $\times$ & $\times$ & $\times$ & $\times$ & $\times$ \\ 
  \quad \R{standardVisual} & Histograms and barplots using ggplot2 &  $\times$ & $\times$ & $\times$ & $\times$ & $\times$ & $\times$ & $\times$ \\ 
  \smallskip \\
 \textbf{\R{checkFunction}s} \smallskip \\
 \quad \R{identifyCaseIssues} & Identify case issues &  $\times$ & $\times$ & & & & &  \\ 
 \quad \R{identifyLoners} & Identify levels with $<$ 6 obs. & $\times$ & $\times$ & & & & &  \\ 
 \quad \R{identifyMissing} & Identify miscoded missing values &  $\times$ & $\times$ & $\times$ & $\times$ & $\times$ & $\times$ &  \\ 
 \quad \R{identifyNums} & Identify misclassified numeric or integer variables & $\times$ & $\times$ & & & & &  \\ 
 \quad \R{identifyOutliers} & Identify outliers &  & & $\times$ & & $\times$ & $\times$ \\ 
 \quad \R{identifyOutliersTBStyle} & Identify outliers (Turkish Boxplot style) &  & & $\times$ & & $\times$ & $\times$ \\ 
 \quad \R{identifyWhitespace} & Identify prefixed and suffixed whitespace &  $\times$ & $\times$ & & $\times$ & & &  \\ 
 \quad \R{isCPR} & Identify Danish CPR numbers & $\times$ & $\times$ & $\times$ & $\times$ & $\times$ & $\times$ &$\times$   \\ 
 \quad \R{isEmpty} & Check if the variable contains only a single value & $\times$ & $\times$ & $\times$ & $\times$ & $\times$ & $\times$ & $\times$  \\ 
 \quad \R{isKey} & Check if the variable is a key & $\times$ & $\times$ & $\times$ & $\times$ & $\times$ & $\times$ & $\times$\smallskip   \\ 
 \hline
\end{tabular}
\caption{List of all summary functions (used in the summary table for
  each variable in the output), visual functions (used for  visualization of each variable), and
  check functions (used for data checks for each variable) currently implemented in \pkg{cleanR}. The variable
  classes C, F, I, L, B, N, and D, refer to \R{character}, \R{factor},
  \R{integer}, \R{labelled}, \R{logical} (boolean), \R{numeric}, and \R{Date}, respectively.}
\label{table.SVCfunctions}
\end{table}


We will now move on to discuss how not only the structure of the
cleaning summary is manipulated, but also its very contents. This is
done through the summarize-visualize-check (SVC) step, as illustrated
in Figure \ref{figure:cleanStructure}.  \pkg{cleanR} uses three
different types of functions for performing these steps, namely
\code{summaryFunction}s, \code{visualFunction}s and
\code{checkFunction}s.  By default, \R{clean()} runs selected summary,
visualization and check functions on each variable in the dataset, and
the exact choice of these functions depends on the classes of the
variables. For instance, though detection of outlier values might be
interesting for numerical variables, it holds little meaning for
factor variables, and therefore, numerical and factor variables need
different checks. Table~\ref{table.SVCfunctions} lists all available
summarize/visualize/check functions, but we can also use the
\R{allSummaryFunctions()}, \R{allVisualFunctions()}, and
\R{allCheckFunctions()} functions in \pkg{cleanR} to print overview
lists in \proglang{R}. For example, the implemented
\R{summaryFunction}s are:

%\subsection{Something about what check, visual and summary functions are available}


\begin{Schunk}
\begin{Sinput}
> allSummaryFunctions()
\end{Sinput}
\begin{Soutput}
----------------------------------------------------------------------------
name           description                     classes                      
-------------- ------------------------------- -----------------------------
centralValue   Compute median or mode          character, Date, factor,     
                                               integer, labelled, logical,  
                                               numeric                      

countMissing   Compute ratio of missing        character, Date, factor,     
               observations                    integer, labelled, logical,  
                                               numeric                      

minMax         Find minimum and maximum        Date, integer, numeric        
               values                                                       

quartiles      Compute 1st and 3rd quartiles   Date, integer, numeric       

uniqueValues   Count number of unique values   character, Date, factor,     
                                               integer, labelled, logical,  
                                               numeric                      

variableType   Data class of variable          character, Date, factor,     
                                               integer, labelled, logical,  
                                               numeric                      
----------------------------------------------------------------------------
\end{Soutput}
\end{Schunk}

Thus we can see, for example, that for \R{numeric}, \R{integer}, or \R{Date} 
variables, \pkg{cleanR} provides functions for adding summary information about 
the minimum and maximum values, while all seven variable
classes dealt with in \pkg{cleanR} have functions for central
tendency summaries (i.e., mode or median).

We can control what summaries and checks are applied for each variable type
through the \R{XXXSummaries} and \R{XXXChecks} arguments, where \R{XXX}
represents a variable type, e.g., \R{factorChecks} for factors,
\R{numericChecks} for numeric variables, etc. These arguments accept a
vector of summary- or check function names that should be run for a
particular variable type. The default values for each of these
arguments can be obtained through the \R{defaultXXXChecks()}
and \R{defaultXXXSummaries()} functions, where \R{XXX} again refers to
the variable type.

For example, the default summaries being used for a factor variable is 
\begin{Schunk}
\begin{Sinput}
> defaultFactorSummaries()
\end{Sinput}
\begin{Soutput}
[1] "variableType" "countMissing" "uniqueValues" "centralValue"
\end{Soutput}
\end{Schunk} 

We can change the summaries (and similarly the checks) by setting the
corresponding arguments when calling \R{clean()}. For example, to get
only the variable type and the central tendency listed in the summary
table of each variable we write

\begin{Schunk}
\begin{Sinput}
> clean(toyData, characterSummaries = c("variableType", "centralValue"),
        factorSummaries = c("variableType", "centralValue"),
        labelledSummaries = c("variableType", "centralValue"),
        numericSummaries = c("variableType", "centralValue"),
        integerSummaries = c("variableType", "centralValue"),
        logicalSummaries = c("variableType", "centralValue"),
        dateSummaries = c("variableType", "centralValue"))
\end{Sinput}
\end{Schunk}

In this particular case, where we specify the same set of summary
functions for each variable type, we can use the simpler argument
\R{allSummaries} which overrides the summary functions for all
variable types. Thus, the same result as above could be obtained with

\begin{Schunk}
\begin{Sinput}
> clean(toyData, allSummaries = c("variableType", "centralValue"))
\end{Sinput}
\end{Schunk}

Similarly, the checks applied are set with the \R{XXXChecks}
arguments. The default checks being applied to a factor is 

\begin{Schunk}
\begin{Sinput}
> defaultFactorChecks()
\end{Sinput}
\begin{Soutput}
[1] "identifyMissing"    "identifyWhitespace" "identifyLoners"    
[4] "identifyCaseIssues" "identifyNums" 
\end{Soutput}
\end{Schunk} 

Now, if we only wanted to apply the identify whitespace function for
factor variables, then we would need to set the \R{factorChecks} accordingly

\begin{Schunk}
\begin{Sinput}
> clean(toyData, factorChecks = c("identifyWhitespace"))
\end{Sinput}
\end{Schunk}

or we could remove checks for factors altogether by setting the
corresponding argument to \R{NULL}, in which case factor variables will
not be checked for any potential errors:

\begin{Schunk}
\begin{Sinput}
> clean(toyData, factorChecks = NULL)
\end{Sinput}
\end{Schunk}

As with \R{summaryFunction}s, a complete list of available
\R{checkFunction}s is obtained by calling
\R{allCheckFunctions()}. Note however, that \R{checkFunction}s have a
usage beyond the \R{XXXChecks} arguments, namely in the
\textit{pre-check} stage. In this stage, it is determined whether or
not each variable is suitable for the summarize/visualize/check (SVC)
steps. The functions used in the pre-check stage should be
\R{checkFunction}s that are applicable to all variable classes. The
default pre-checks, the functions \R{isKey()} and \R{isEmpty()}, check
whether a variable has unique values for all observations or only a
single value for all observations, respectively. If one of these
statements are true, the variable will not be subjected to the SVC
steps.  We can allow empty variables to move on to the SVC step by
only checking for keys in the pre-check step:

\begin{Schunk}
\begin{Sinput}
> clean(toyData, preChecks = "isKey")
\end{Sinput}
\end{Schunk}

Note that the data visualizations in the cleaning summary are also
controllable, though only a single function can be provided for all
variable types. If for instance, we wish to change the visualizations
from the default \pkg{ggplot2} style histograms and barplots to base
\proglang{R} histograms and barplots, we type
\begin{Schunk}
\begin{Sinput}
> clean(toyData, allVisuals = "basicVisual")
\end{Sinput}
\end{Schunk}


%\hl{
%\begin{itemize}
%\item Introduce the relevant \R{clean}-arguments and the \R{defaultWhateverSummaries} etc.-functions
%\item Introduce \R{allSummaryFunctions()} etc. and present a table corresponding to the output of this call
%\item Small examples: 
%\begin{itemize}
%\item Add a function to one of the XXXSummaries/XXXVisuals/XXXChecks-arguments (still calling default options)
%\item Remove all but a single function from one of these arguments
%\item Describe what happens if the argument is NULL
%\end{itemize}
%\end{itemize}
%}





In summary, and as indicated in Figure~\ref{figure:cleanStructure}, there are two stages
where \R{clean()} applies functions to each of the variables: 
\begin{enumerate}
\item In the pre-check stage
\item As part of the summarize/visualize/check (SVC) steps
\end{enumerate}
Each of these stages are controllable using appropriate function
arguments in \R{clean()}, and above we have shown examples of how to
tweak them to modify the data cleaning outputs. However, if for
instance the dataset at hand requires new, additional checks, then more control is needed, and
Section~\ref{sec:extending} explains how to modify and expand the
possibilities by producing new summary-, visual- and check functions.

\section{Using cleanR interactively}
\label{sec:interactiveCleanR}

While overview documents are great for presenting and documenting the
data cleaning checks, it may be useful to be able to work more interactively
through the data cleaning process as well. Aside from the \R{clean()} function 
presented above, \pkg{cleanR} also provides more standard \proglang{R} interactive tools, such as
functions that print results to the console or returns the information
as an object for later use. This section describes how to use the
functions \R{check()}, \R{summarize()} and \R{visualize()} to work
interactively with \pkg{cleanR}. 

\subsection{Data cleaning by hand: An example}
Let's say we wish to look further into a certain variable from
\R{toyData}, namely \R{var2}. The data cleaning summary found some
issues in this variable, and we would like to recall what these issues
were. This can be done using the \R{check()} command
\begin{Schunk} 
\begin{Sinput} 
> check(toyData$var2)
\end{Sinput}
\begin{Soutput}
$identifyMissing
The following suspected missing value codes enter as regular values: 999, NaN.
$identifyOutliers
Note that the following possible outlier values were detected: 82, 999. 
\end{Soutput}
\end{Schunk}
% $

Note that the arguments specifying which checks to perform, as
described in the previous section, are in fact passed to \R{check()},
and thus they can also be used here. For instance, if we only want to
check for potentially miscoded missing values, we can use the relevant
\R{XXXChecks} argument (e.g., \R{numericChecks}, \R{factorChecks},
etc. as described in Section~\ref{sec:usingcleanR}) to provide a
vector of the check functions that should be applied. Recall that
Table~\ref{table.SVCfunctions} or an \R{allCheckFunctions()} call provide
overviews of the available check functions. 
Moving forward, we limit the numeric checks to only identify miscoded 
missing values:

\begin{Schunk}
\begin{Sinput}
> check(toyData$var2, numericChecks = "identifyMissing")
\end{Sinput}
\begin{Soutput}
$identifyMissing
The following suspected missing value codes enter as regular values: 999, NaN.
\end{Soutput}
\end{Schunk}

An equivalent way to call only a single, specific \R{checkFunction},
such as \R{identifyMissing}, is by using it directly on the variable, e.g.,
\begin{Schunk}
\begin{Sinput}
> identifyMissing(toyData$var2)
\end{Sinput}
\begin{Soutput}
The following suspected missing value codes enter as regular values: 999, NaN. 
\end{Soutput}
\end{Schunk}
% $


The result of a \R{checkFunction} is an object of class
\R{checkResult}. By using the structure function, \R{str()}, we can
look further into its components:

\begin{Schunk}
\begin{Sinput}
> missVar2 <- identifyMissing(toyData$var2)
> str(missVar2)
\end{Sinput}
\begin{Soutput}
List of 3
 $ problem      : logi TRUE
 $ message      : chr "The following suspected missing value codes enter as 
 		 regular values: \\\"999\\\", \\\"NaN\\\"."
 $ problemValues: num [1:2] 999 NaN
 - attr(*, "class")= chr "checkResult"
\end{Soutput}
\end{Schunk}
The most important thing to note here is that while the printed
message is made for easy reading, the actual values of the variable
causing the issue are still obtainable in the element
\R{problemValues}. If we for instance decide that the values \R{999}
and \R{NaN} in \R{var2} are in fact miscoded missing values, we can
easily replace them with \R{NA}s:
\begin{Schunk}
\begin{Sinput}
> toyData$var2[toyData$var2 %in% missVar2$problemValues] <- NA
> identifyMissing(toyData$var2}
\end{Sinput}
\begin{Soutput}
No problems found.
\end{Soutput}
\end{Schunk}
% $

Similarly, the \R{visualize()} and \R{summarize()} functions can be
used to run the corresponding visualizations and summaries for each
variable. See Figure~\ref{fig:example2} for the visualization output.

\begin{Schunk}
\begin{Sinput}
> visualize(toyData$var2)
> summarize(toyData$var2)
\end{Sinput}
\begin{Soutput}
     Feature                   Result    
[1,] "Variable type"           "numeric" 
[2,] "Number of missing obs."  "3 (20 %)"
[3,] "Number of unique values" "8"       
[4,] "Median"                  "4.5"     
[5,] "1st and 3rd quartiles"   "1.75; 6" 
[6,] "Min. and max."           "1; 999"  
\end{Soutput}
\end{Schunk}
% $

\begin{figure}[tb]
\begin{center}
\includegraphics[width=7.5cm]{toyData-var2.pdf}
\end{center}
\label{fig:example2}
\caption{Output from running \R{visualize()} on the variable \texttt{var2} from the
\R{toyData} dataset.}
\end{figure}


As we saw with the \R{check()} function, the summary can be modified
by providing the relevant \R{XXXSummaries} arguments. Setting the
\R{numericSummaries} argument, we can control the summary output by
providing a vector of function names to run for a particular summary. To
only get the median, minimum and the maximum we set
\R{numericSummaries=c("centralValue", "minMax")}:

\begin{Schunk}
\begin{Sinput}
> summarize(toyData$var2, numericSummaries = c("centralValue", "minMax"))
\end{Sinput}
\begin{Soutput}
     Feature         Result  
[1,] "Median"        "4.5"   
[2,] "Min. and max." "1; 999"
\end{Soutput}
\end{Schunk}
% $

Note that the \R{summarize()}, \R{check()} and \R{visualize()} functions are also available interactively for full datasets, by calling e.g. \R{summarize(toyData)}. However, this produces an extensive amount of output in the console, and therefore, we generally do not recommend it, unless working with very small datasets or subsets of datasets.

%\hl{
%\begin{itemize}
%\item Do an example with visualize() and summarize(), like the one with check(). Especially visualize and to doEval = T thing needs a bit of special attention.
% \item  Mention allCheckFunctions() etc. again here
% \item Mention check(), visualize() and summarize() modes for data.frames. Maybe also advice against it, at it will often produce a lot of information at once, and such large amounts of information really should be documented.
%\end{itemize}
%}

\section{Extending cleanR} 
\label{sec:extending}

\pkg{cleanR} can be used as a user-friendly, self-contained package, as
shown in the previous sections. However, \pkg{cleanR} is fully
customizable and \R{clean()} is mainly a tool for formatting the
results from various checking-, summary- and visualization
functions. Thus, the actual work underlying a \pkg{cleanR} output file
can be anything --- depending on the arguments given to \R{clean()}
--- and user made functions are easily added to the
summarize/visualize/check-function arguments, as discussed previously. 
However, the functions used in the SVC steps must adhere
to specific structures in order to be called from these three steps, and
therefore, we will now present how \R{summaryFunction}s,
\R{visualFunction}s and \R{checkFunction}s are made. 


This section consists of two parts. First, we describe how customized \R{summaryFunction}s, 
\R{visualFunction}s and \R{checkFunctions} can be made, including an introduction to a 
few convenient tools available in \pkg{cleanR} for aiding SVC-function construction. 
Table \ref{table.functionTypes} serves as an overview of the internals of these three
central function types. Secondly, we turn to a 
worked example of how to use custom made functions in practice. Here,
four new SVC functions are defined and used, both interactively and in \R{clean()}.

\begin{table}[htbp]
\footnotesize
\bgroup
\def\arraystretch{1.8}% 
\begin{tabular}{L{0.2\linewidth}L{0.2\linewidth}L{0.2\linewidth}L{0.2\linewidth}}
& \R{summaryFunction} & \R{visualFunction} & \R{checkFunction} \\
\hline
\vspace{0pt} Input (required) & \R{v} - a variable vector \newline \R{...}  &  \R{v} - a variable vector \newline \R{vnam} - the variable name (as character string) \newline \R{doEval} - a logical (\R{TRUE}/\R{FALSE}) controlling the output type of the function & \R{v} - a variable vector \newline \R{nMax} - an integer (or \R{Inf}), controlling how many problematic values are printed, if relevant \newline \R{...}  \\
Input (optional) &  \R{maxDecimals} - number of decimals printed in outputted numerical values.  & - &  \R{maxDecimals}  - number of decimals printed in outputted numerical values.  \\
Purpose & Describe some aspect of the variable, e.g. a central value, its dispersion or level of missingness. & Produce a distribution plot. & Check a variable for a specific issue and, if relevant, identify the values in the variable that cause the issue. \\
Output (required) & A list with entries: \newline \R{\$feature} - a label for the summary value (as character string); \newline \R{\$result} - the result of the summary (as character string) & A character string with \R{R} code for producing a plot. This code should be standalone, i.e. should include the data if necessary. & A list with entries: \newline \R{\$problem} - a logical identifying whether an issue was found; \newline \R{\$message} - a character string (possibly empty) decribing the issue that was found, properly escaped and ready for use in \R{rmarkdown} \\
Output (recommended) & A \R{summaryResult} object, i.e. an attributed list with entries \R{\$feature}, \R{\$result} and \R{\$value}, the latter being the values from \R{\$result} in their original format). & \bgroup
\def\arraystretch{1.8}% 
\hspace*{-.2cm}\begin{tabular}[t]{p{0.45\linewidth} | p{0.45\linewidth}} \vspace*{-1cm}If \R{doEval} is \R{TRUE}: \newline A plot that is opened by the graphic device in \R{R}. & \vspace*{-1cm}If \R{doEval} is \R{FALSE}: \newline   A text string with \R{R} code, as described above. \end{tabular} \egroup  & A \R{checkResult} object, i.e. an attributed list with entries \R{\$problem}, \R{\$message} and \R{\$problemValues}, the latter being either \R{NULL} or the problem causing values, as they were found in \R{v}, whichever is relevant. \\
Tools available for producing the function & \R{summaryResult()} & --- & \R{messageGenerator()} \newline \R{checkResult()} 
\end{tabular}
\egroup
\caption{Reference information for creating new functions to be used
  as part of the \mbox{summarize-,} visualize-, and check steps. The three
  columns correspond to each of the three function types. %\hl{I think the formatting here is still sort of chaotic. Maybe include horizontal lines everywhere? Is it allowed in JSS? It's impossible to read what's in which row right now.}
}
\label{table.functionTypes}
\end{table}



\subsection{Function templates}
\label{sec:functionTemplates}
In order to construct a summary-, visual or check function, one needs
to create a new function with a specific structure.  This can be done with 
different levels of strictness. If
the new custom function is only to be used as part of the SVC steps in \R{clean()},
then only the input/output structure of the function needs special attention.
However, new user-defined functions can also be registered locally to
be part of the full machinery of \pkg{cleanR}, and these function will
be recognized and behave in the same way as the build-in functions in
\pkg{cleanR}. The presentation below is given in the format of
function templates, written in pseudo-code. These templates are
designed for getting the full functionality, but note that
Table~\ref{table.functionTypes} serves as a reference to the minimal
requirements, while also presenting the "full" versions of the
function types. 

% \hl{metatext, mention Table \ref{table.functionTypes} again.}

\subsubsection{Writing a summaryFunction}
As mentioned above, \pkg{cleanR} provides a dedicated class for
\code{summaryFunction}s. However, this does not imply that they are
particularly advanced or complicated to create; in fact, they are
nothing but regular functions with a particular
input/output-structure. Specifically, they all follow the template
below:
\begin{Verbatim}
mySummaryFunction <- function(v, ...) {
  res <- [result of whatever summary we are doing]
  summaryResult(list(feature = "[Feature name]", result = res))
}
\end{Verbatim}
The last function called here, \R{summaryResult()}, changes the class
of the output, thereby making a \R{print()} method available for it.
Note that \code{v} is a vector and that \code{res} should be either a
character string or something that will be printed as one. In other
words, e.g. integers are allowed, but matrices are not. Though a lot
of different things can go into the \code{summaryFunction} template,
we recommend only using it for summarizing the features of a variable,
and leaving tests and checks for the \code{checkFunction}s (presented
below).

Adhering to the template above is sufficient for using the freshly
made \R{mySummaryFunction()} in \R{clean()}, but we recommend
furthermore adding the new function to the overview of all summary
functions by converting it to a proper \code{summaryFunction}
object. This is done by calling the \R{summaryFunction()} creator with
the user-defined function as the first argument, and additional arguments
\R{description} (an explanatory text which will be added to the attributes of
the function), and \R{classes} (a vector of variable classes the
user-defined function is intended to be applied to, also stored as an
attribute). In other words, a call on the following form should be made:
\begin{Schunk}
\begin{Sinput}
mySummaryFunction <- summaryFunction(mySummaryFunction,
  description = "[Text describing what the summaryFunction does]",
  classes = c([vector of data types that the function is intended for]))
\end{Sinput}
\end{Schunk}
which adds the new function to the output of an
\code{allSummaryFunctions()} call. 
% One comment should be devoted to
% the two attributes of a \R{summaryFunction}. 
If the \R{description} argument is left unspecified, the name of the
function (which in this case is \R{"mySummaryFunction"}) will be filled in and
stored under the \R{description} attribute. What happens if the \R{classes} argument is
not specified depends on the type of \R{mySummaryFunction}. If
\R{mySummaryFunction} is constructed as an \R{S3} generic function with associated
methods, the call to \R{summaryFunction()} will automatically produce
a vector of the names of the classes for which the function can be
called. If \R{mySummaryFunction} is not an \R{S3} generic and
\R{classes} is left unspecified, the attribute will simply be
left empty. Note that the helper function \R{allClasses()} might be useful
for filling out the \R{classes} argument, as it simply lists all
available classes in \pkg{cleanR}:
\begin{Schunk}
\begin{Sinput}
> allClasses()
\end{Sinput}
\begin{Soutput}
[1] "character" "Date"      "factor"   "integer"   "labelled"  
[6] "logical"  "numeric"  
\end{Soutput}
\end{Schunk}
% \hl{Write something here, don't end paragraph with code. Also, maybe move the allClasses() stuff somewhere else, it doesn't really belong under this header. Not sure where to, though.}


%Jeg foreslår at droppe nedenstående, da jeg synes det bliver lidt søgt i forhold 
%til at det kun er beskrevet i pseudokode. (fx. er classes = numeric meget arbitrært)
%If we run the \R{allSummaryFunctions()} function we can see that the
%new function has been registered (output slightly appreviated):

%\begin{Schunk}
%\begin{Sinput}
%> allSummaryFunctions()
%\end{Sinput}
%\begin{Soutput}
%----------------------------------------------------------------------------
%name                description                                      classes
%------------------- --------------------------------------------------------
%mySummaryFunction   [Text describing what the summaryFunction does]  numeric
%.
%.
%.
%\end{Soutput}
%\end{Schunk}



\subsubsection{Writing a visualFunction}
\code{visualFunction}s are the functions that produce the figures in a
\pkg{cleanR} output document. Writing a \R{visualFunction} is slightly
more complicated than writing a \R{summaryFunction}. This follows from
the fact that \R{visualFunctions} need to be able to output standalone
code for plots in order for \code{clean()} to build standalone
\pkg{rmarkdown} files. We recommend using the following structure:
\begin{Schunk}
\begin{Sinput}
myVisualFunction <- function(v, vnam, doEval) {
  thisCall <- call("[the name of the function used to produce the plot]",
    v, [additional arguments for the plotting function])
  if (doEval) {
    return(eval(thisCall))
  } else return(deparse(thisCall))
}
\end{Sinput}
\end{Schunk}

In this function, \code{v} is the variable to be visualized,
\code{vnam} is its name (which should generally be passed to
\code{title} or \code{main} arguments in plotting functions) and
\code{doEval} controls whether the output is a plot (if \code{TRUE})
or a character string of standalone code for producing a plot (if
\code{FALSE}). Implementing the \code{doEval = TRUE} setting is not
strictly necessary for a \R{visualFunction}'s use in \code{clean()}, but
it makes it easier to assess what visualization options are available,
and obviously, it is crucial for interactive usage of
\R{myVisualFunction()}. In either case, it should be noted that all
the parameters listed above, \code{v}, \code{vnam} and \code{doEval},
are mandatory, so they must be left as is, even if they are not in
use (c.f. Table~\ref{table.functionTypes}).


As with the summary function, we call \R{visualFunction()} to
register our newly created function:
%. Like before, it accepts the
%user-defined function as the first argument, and two additional
%arguments description which adds explanatory flavour text as an
%attribute of the function, and a vector which determines which classes
%the user-defined function should be applied to (which is also stored
%as an attribute).
\begin{Schunk}
\begin{Sinput}
myVisualFunction <- visualFunction(myVisualFunction,
  description = "[Some text describing the visualFunction]",
  classes = c([data types that this function is intended for]))
  )
\end{Sinput}
\end{Schunk}

Now, \R{myVisualFunction()} will be available in a \R{allVisualFunctions()}
call, just like the two build-in \R{visualFunction}s, \R{standardVisual}
and \R{basicVisual}. 

\subsubsection{Writing a checkFunction}
The last, but perhaps most important, \pkg{cleanR} function type is
the \code{checkFunction}. These are the functions that flag issues in
the data in the check step and control the overall flow of the data
cleaning process in the precheck stage. A \code{checkFunction} follows
one of two overall structures, depending on the type of check. Either,
it tries to identify problematic values in the variable (as
e.g., \R{identifyMissing()} does), or it performs a check concerning the
variable as a whole (e.g. the functions used for prechecks and the
function \R{identifyNums()}). We present templates for both types of
\R{checkFunction}s below separately, but it should be emphasized that
formally, they belong to the same class.

First, a template for the full-variable check function type, where we
first define the function and subsequently register it as a check
function using \R{checkFunction()}:
\begin{Schunk}
\begin{Sinput}
myFullVarCheckFunction <- function(v, ...) {
  [do your check]
  problem <- [is there a problem? TRUE/FALSE]
  message <- "[message describing the problem, if any]"
  checkResult(list(problem = problem,
    message = message,
    problemValues = NULL))
}

myFullVarCheckFunction <- checkFunction(myFullVarCheckFunction, 
  description = "[Some text describing the checkFunction]",
  classes = c([the data types that this function is intended to be used for])
  )
\end{Sinput}
\end{Schunk}

Again, as with \R{summaryFunction}s and \R{visualFunction}s, the
change of function class by use of \R{checkFunction()} is not strictly
necessary. Note however, that if \R{myFullVarCheckFunction} is to be
used in the summarize/visualize/check steps in \R{clean()}, the
description attribute will be printed in the overview table in the
\textit{Data cleaning summary} part of the output document.

If problematic values are to be identified, the template from above
should be expanded to follow a slightly more complicated structure:
\begin{Schunk}
\begin{Sinput}
myProbValCheckFunction <- function(v, nMax, maxDecimals, ...) {
	[do your check]
	problem <- [is there a problem? TRUE/FALSE]
	problemValues <- [vector of values in v that are problematic]
	problemStatus <- list(problem = problem, 
	                      problemValues = problemValues)

	problemMessage <- "[Message that is printed prior to listing
			    problem values in the cleanR output, 
                            ending with a colon]"

	outMessage <- messageGenerator(problemStatus, problemMessage, nMax)

	checkResult(list(problem = problem,
	                 message = outMessage,
	                 problemValues = problemValues)) 
}

myProbValCheckFunction <- checkFunction(myProbValCheckFunction,
  description = "[Some text describing the checkFunction]",
  classes = c([the data types that this function is intended to be used for])
)
\end{Sinput}
\end{Schunk}
One comment should be devoted to the helper function,
\R{messageGenerator()}.  This function's sole purpose is aiding
consistent styling of all \code{checkFunction} messages. The function
simply pastes together the \code{problemMessage} and the
\code{problemValues}, with the latter being quoted and sorted
alphabetically. If the \R{nMax} argument to \R{messageGenerator()} is
not \R{Inf}, only the first \R{nMax} problem values will be pasted
onto the message, accompanied by a comment about how many problem
values were left out (if any).  Note that printing quotes in
\pkg{rmarkdown} requires an extensive amount of character escaping, so
opting for \code{messageGenerator()} really is the easiest solution.

In the template above, the argument \R{maxDecimals} is not in use. This
argument should be used to round off the \R{problemValues} passed to
\R{messageGenerator()}, if they are numerical.  This can be done by
adding an extra line of code after defining \R{problemStatus} in the template 
above:
\begin{Schunk}
\begin{Sinput}
myProbValCheckFunction <- function(v, nMax, maxDecimals, ...) {
 ...
  problemStatus <- list(problem = problem, 
    problemValues = problemValues)
  
  if (!is.null(problemValues)) {
    problemStatus$problemValues <- round(problemValues, maxDecimals)
  }
 ...  
}
\end{Sinput}
\end{Schunk}
%$
Now, problematic values will be rounded in the outputted message, while they
will still appear in their original format under the entry \R{\$problemValues} %$
in the outputted \R{checkResult}.

\subsection{A worked example}

We will now build four new functions and show both how they can be
used interactively and how they can be integrated with the \R{clean()}
function. These four new functions are:
\begin{description}
\item[\R{isID}] A new \R{checkFunction} intended for use in the precheck-stage. This function checks whether a variable consists exclusively of long ($> 10$ characters/digits) entries that are all of equal length, as this might be personal identification codes that we do not wish to print out in the data summary.  
\item[\R{mosaicVisual}] A new \R{visualFunction} that produces mosaic plots. This function will be used in the \textit{visualize} step of \R{clean()}.
\item[\R{countZeros}] A new \R{summaryFunction} that counts the number of occurrences of the value \R{0} in a variable. This function will be used in the \textit{summarize} step of \R{clean()}.
\item[\R{identifyColons}] A new \R{checkFunction} that flags variables in which values have colons that appear before and after alphanumerical characters. This is practical for identifying autogenerated interaction effects. This function will be used in the \textit{check} step of \R{clean()}. 
\end{description}
These functions are defined in turn below, and afterwards, an example of how they can be called from \R{clean()} is provided.

\subsubsection{isID --- a new checkFunction without problem values}
First, let's define the \R{isID()} function. As this function is not
supposed to list problematic values, it falls within
the category of \R{checkFunctions} represented by
\R{myFullVarCheckFunction()} in the above. We do not particularly wish
to use this function interactively, so we will stick to the minimal
requirements of a \R{checkFunction} used in \R{check()} (see
Table~\ref{table.functionTypes}). The function can then be defined by
\begin{Schunk}
\begin{Sinput}
isID <- function(v, nMax = NULL, ...) {
  out <- list(problem = FALSE, message = "")
  # Should work for all but logical and Date variables
  if (class(v) %in% setdiff(allClasses(), c("logical", "Date"))) {
    v <- as.character(v)
    lengths <- c(nchar(v))
    # Check that all lengths are the same and greater than 10 characters
    if (all(lengths > 10) & length(unique(lengths)) == 1) {
      out$problem <- TRUE
      out$message <- "Warning: This variable seems to contain ID codes."
    }
  }
  out
}
\end{Sinput}
\end{Schunk} 

This is essentially all we need to do in order to include this
function as a precheck-function in \R{clean()}, so we will leave it as
is and move on to the next function, namely \R{mosaicVisual()}.

\subsubsection{mosiacVisual --- a new visualFunction}
\proglang{R} contains a function, \R{mosiacplot()}, which produces
mosaic plots. We intend to use this existing high-level plotting
function as part of the new visualization function.  We will define
the new function such that it gets the full \pkg{cleanR}
functionality. This can be done using the following code.
\begin{Schunk}
\begin{Sinput}
 mosaicVisual <- function(v, vnam, doEval) {
   # Setup the call using the existing function mosaicplot
   thisCall <- call("mosaicplot", table(v), main = vnam, xlab = "")
   if (doEval) {                     # Return the graphics
    return(eval(thisCall))
   } else return(deparse(thisCall))  # Else return the code for the call
 }
\end{Sinput}
\end{Schunk}
This function can now be called directly or used in \R{clean()}, as will
presented in an example below. Depending on the \R{doEval} argument, either a text string
with code or a plot is produced. The plot resulting from the following
call is found in Figure~\ref{fig:mosaicPlot}:
\begin{Schunk}
\begin{Sinput}
> mosaicVisual(toyData$var1, "variable 1", doEval = TRUE)   
\end{Sinput}
\end{Schunk}
% $

\begin{figure}
\begin{center}
\includegraphics[scale=0.2]{mosaicPlotExample.pdf}
\end{center}

\caption{Mosaic plot generated by the user-defined visualization
  function \texttt{mosaicVisual}.}
\label{fig:mosaicPlot}
\end{figure}




Even though \R{mosaicVisual()}, as written above, follows the style of a
\R{visualFunction}, it is not yet truly one and therefore, it will not
appear in an \R{allVisualFunctions()} call. In order to get this
functionality, we need to change its object class. This can be done by
writing
\begin{Schunk}
\begin{Sinput}
 mosaicVisual <- visualFunction(mosaicVisual, 
 			description = "Mosaic plots using graphics",
                                classes = allClasses())
\end{Sinput}
\end{Schunk}
Here, we use the function \R{allClasses()} to quickly obtain a vector
of all the seven variable classes addressed in \pkg{cleanR}. Note that
if \R{mosaicVisual()} were an S3 generic function, this argument could
have been left as \R{NULL} and then the classes for which methods are
available would be added automatically. 
%\hl{I'm repeating myself here,
%  but I think it is quite a neat feature, so maybe that's okay?}

As \R{mosaicVisual} is now a full-blooded \R{visualFunction}, it will also be included in the \R{allVisualFunctions()} output table:
\begin{Schunk}
\begin{Sinput}
> allVisualFunctions()
\end{Sinput}
\begin{Soutput}
------------------------------------------------------------------------
name           description                   classes                    
-------------- ----------------------------- ---------------------------
mosaicVisual   Mosaic plots using graphics   character, Date, factor,   
                                             integer, labelled, logical,
                                             numeric                    

basicVisual    Histograms and barplots using character, Date, factor,   
               graphics                      integer, labelled, logical,                           
                                             numeric                    
standardVisual Histograms and barplots using character, Date, factor,                              
               ggplot2                       integer, labelled, logical,               
                                             numeric                    
------------------------------------------------------------------------
\end{Soutput}
\end{Schunk}

Now that we are done with the definition of \R{mosaicVisual()} we can
turn to the next function in line, \R{countZeros}.

\subsubsection{countZeros --- a new summaryFunction}
This \R{summaryFunction} is defined in the following lines of code:
\begin{Schunk}
\begin{Sinput}
countZeros <- function(v, ...) {
	res <- length(which(v == 0))
	summaryResult(list(feature = "No. zeros", result = res, value = res))
}
\end{Sinput}
\end{Schunk}
As this function computes an integer (the number of zeros),
there is no difference between the entries \R{\$result} and
\R{\$value}. If, on the other hand, the result had been a character
string, extra formatting might be required in the \R{\$result} entry
(such as escaping of quotation marks), and in this scenario, the two
entries would have differed. As the result is returned as a
\R{summaryResult} object, a printing method is automatically called
when \R{countZeros} is used interactively:
\begin{Schunk}
\begin{Sinput}
> countZeros(c(rep(0, 5), 1:100))
\end{Sinput}
\begin{Soutput}
No. zeros: 5
\end{Soutput}
\end{Schunk}
As with \R{mosaicVisual()}, we change the class of this function in
order to make it appear in \R{allSummaryFunctions()} calls. But now we
wish to emphasize that the function is not intended to be called on
all variable types, as zeros have different roles in \R{Date}s and in
\R{logical} variables:
\begin{Schunk}
\begin{Sinput}
> countZeros <- summaryFunction(countZeros,
	description = "Count number of zeros",
	classes = c("character", "factor", "integer", 
                    "labelled", "numeric"))
\end{Sinput}
\end{Schunk}
% \hl{more? don't end on code.}

\subsubsection{identifyColons - a new checkFunction with problem values}
The last function mentioned above is \R{identifyColons()}. We define
it using the helper function \R{messageGenerator()} to obtain a properly
escaped message, and we use \R{checkResult} to make its output print
neatly:
\begin{Schunk}
\begin{Sinput}
identifyColons <- function(v, nMax = Inf, ... ) {
  v <- unique(na.omit(v))
  problemMessage <- "Note: The following values include colons:"
  problem <- FALSE
  problemValues <- NULL

  # Use regular expressions to identify colons between two words  
  problemValues <- v[sapply(gregexpr("[[:xdigit:]]:[[:xdigit:]]", v),
                            function(x) all(x != -1))]
  
  if (length(problemValues) > 0) {
    problem <- TRUE 
  }
  
  problemStatus <- list(problem = problem, 
                        problemValues = problemValues)
  # Use the messagegenerator function to produce the output 
  # message from the problemStatus and problemMessage
  outMessage <- messageGenerator(problemStatus, problemMessage, nMax)
  
  checkResult(list(problem = problem, 
                   message = outMessage,
                   problemValues = problemValues))
}

identifyColons <- checkFunction(identifyColons, 
    description = "Identify colons surrounded by alphanumeric characters",
    classes = c("character", "factor", "labelled"))
\end{Sinput}
\end{Schunk}
As with the previous two functions, we also change its class. Note,
however, that for \R{checkFunction}s, the function description will
appear in the document produced by \R{clean()} (in the \textit{Data
  cleaning summary} section of the output), so now this is not only
done for the sake of the \R{allCheckFunctions()} output.

\subsubsection{Calling the new summarize/visualize/check functions from clean()}
Now, we are ready to use these new functions in a \R{clean()}
call. The extended \pkg{cleanR} output document should have the
following modifications, relative to the standard \pkg{cleanR} output:
\begin{itemize}
\item We want to add the new pre-check function, \R{isID}, to the already existing pre-checks.
\item We wish to change the plot type for all variables to the new mosaic plot.
\item We want the new summary function, \R{countZeros}, to be added to the summaries performed on all variable types but \R{Date} and \R{logical}.
\item We want the new check function, \R{identifyColon}, to be added to the checks performed on \R{character}, \R{factor} and \R{labelled} variables.
\end{itemize}
These options are specified as follows: %\hl{Jeg synes ikke testData egner sig her. Der er ingen koloner og CPRvar er et mærkeligt, internt navn, hvis ikke man kender danske CPR-numre. Langt de fleste variable har desuden ingen nuller. Måske skal der laves et nyt datasæt?}
\begin{Schunk}
\begin{Sinput}
> data(exampleData) 
> clean(exampleData, 
      # Add the new pre-check function
      preChecks = c("isKey", "isEmpty", "isID"),
      # Change the visual
      allVisuals = "mosaicVisual",
      # Add the new summary for specified data types
      characterSummaries = c(defaultCharacterSummaries(), "countZeros"),
      factorSummaries = c(defaultFactorSummaries(), "countZeros"),
      labelledSummaries = c(defaultLabelledSummaries(), "countZeros"),
      numericSummaries = c(defaultNumericSummaries(), "countZeros"),
      integerSummaries = c(defaultIntegerSummaries(), "countZeros"),
      # Add the new check for specific data types
      characterChecks = c(defaultCharacterChecks(), "identifyColons"),
      factorChecks = c(defaultFactorChecks(), "identifyColons"),
      labelledCheck = c(defaultLabelledChecks(), "identifyColons"))
\end{Sinput}
\end{Schunk}
The outputted document is found in Appendix \ref{sec:appendix2}. 

\section{Rubbing down data cleaning challenges}
\label{sec:specificExamples}
%\hl{Lidt poppet titel, men den gamle ("Using cleanR in specific situations")
%fik mig til flabel at tænke "i modsætning til uspecifikke situationer?". Jeg 
%er dog meget åben for en bedre titel.}

Finally, we present a few examples of how to make \pkg{cleanR}
solve specific issues related to data cleaning. First, we discuss the
challenges related to cleaning large datasets, particularly in terms
of memory use and computation speed. Next, we show how \pkg{cleanR}
can be used for problem-flagging. Lastly, we discuss how the
\pkg{cleanR} output document can be included in other \pkg{rmarkdown}
documents as a way to produce clear and concise documentation of a
dataset. %\hl{I feel like there should be more topics here, but I'm all
%  out of ideas...}

\subsection{Cleaning large datasets}
If the dataset becomes very large, the standard use of \R{clean()}
outlined above might not be ideal. If there is a vast number of
variables, creation of the \pkg{rmarkdown} document might be quite
slow, while a large number of observations will generally affect the
rendering time of the document. In this section, we give a few
practical examples of ways to deal with large data, while wishing to
still produce (potentially very long) data cleaning overview
documents. Note that the interactive tools of \pkg{cleanR} can be used
as usual or sequentially in small subsets of the large dataset, if no
such overview documents are needed.

\subsubsection{Handling the figures}
Though figures give a nice overview of each variable, they are also
quite heavy objects in terms of memory allocation. Therefore, it might
be beneficial to not include figures in the \pkg{cleanR} outputs for
very large datasets. This is controlled via the \R{mode} argument:
\begin{Schunk}
\begin{Sinput}
> clean(toyData, mode = c("summarize", "check"))
\end{Sinput}
\end{Schunk}
If figures are indeed needed, a different approach is to choose the
less memory heavy standard \proglang{R} figure style instead of the
\pkg{ggplot2} figures that are the default option in \R{clean()}. This
can be done by setting the \R{allVisuals = "basicVisual"} argument:
\begin{Schunk}
\begin{Sinput}
> clean(toyData, allVisuals = "basicVisual")
\end{Sinput}
\end{Schunk}
Of course, even less heavy plots might be achieved by writing new
\R{visualFunction}s, using the guidelines from section
\ref{sec:functionTemplates}. For instance, a future extension of
\pkg{cleanR} might be the inclusion of ASCII plots, as
e.g. represented in the \proglang{R} package \pkg{txtplot}.

%\hl{I really feel like we should do some benchmarking here, maybe just on toyData, both in terms of speed and memory use. I would make the recommendations more trustworthy and serious.}

\subsubsection{Economic memory use} 
Another solution, which is especially relevant to Windows users due to
the unfortunate combination of memory control in this operating system
and RStudio, \hl{And also just R, right? Du har udkommenterert denne kommentar,
men jeg mener stadig at det er tilfældet, og jeg mener desuden ikke vi kan skrive 
det uden en præcision af hvad det går ud på. Hvad tænker du?}
is simply splitting the two steps performed
by \R{clean()}, namely producing the \pkg{rmarkdown} file and rendering it
afterwards. If the \R{rmarkdown} file is very long, as it will
typically be for very large datasets, keeping this file open in memory
wastes precious memory capacities. Therefore, we advice users to
instead split the two steps. This can be done in the following manner:

\begin{Schunk}
\begin{Sinput}
> clean(toyData, render = FALSE, openResult = FALSE}
> render("cleanR_toyData.Rmd", quiet = FALSE)
\end{Sinput}
\end{Schunk}
This also deals with the fact that \pkg{cleanR} can produce
\pkg{rmarkdown} files that supersedes the upper size limit. \hl{dette giver
ingen mening uden en præcision? "upper size limit" af hvad? Se udkommenteret 
kommentar.}
% of RStudio,
%which is currently \hl{find number} GBs (using RStudio version
%1.0.44). 
%\hl{Is this maybe too editor specific? On the other hand, a
%  lot of people do use RStudio...}.

\subsection{Using cleanR for problem flagging}
If the data is large, but memory issues and computation time are less
of an issue than the human time it takes to look through the data
cleaning document, a viable solution might be not to include all
information about all variables. Or even for more reasonably sized
datasets, sometimes a brief overview of the most pressing issues can
be useful. This can be achieved by using the \R{onlyProblematic}
argument in \R{clean()}. By specifying \R{onlyProblematic = TRUE},
only variables that raise a flag in the checking steps will be
summarized and visualized. But perhaps we are not even interested in
obtaining general information about these variables, but only in
getting a quick overview of the problems they might have. This can be
done by also controlling the \R{mode} argument:
\begin{Schunk}
\begin{Sinput}
> clean(toyData, onlyProblematic = TRUE, mode = c("check"))
\end{Sinput}
\end{Schunk}
Now only the checking results are printed, and only for variables where problems were identified. An even more minimal output can be generated by also leaving out the checking results --- then \R{clean()} essentially just produces a list of the variable names that should be investigated further:
\begin{Schunk}
\begin{Sinput}
> clean(toyData, onlyProblematic = TRUE, mode = NULL)
\end{Sinput}
\end{Schunk}
Of course, this can also be done without generating an overview
document, by direct, interactive use of the \R{check()} function. When
called on a \R{data.frame}, this function produces a list (of
variables) of lists (of checks) of lists (or rather,
\R{checkResult}s). Thus, the overall problem status of each variable
can easily be unravelled using the list manipulation function
\R{sapply()}:
\begin{Schunk}
\begin{Sinput}
> toyChecks <- check(toyData)
> foo <- function(x) {
>   any(sapply(x, function(y) y[["problem"]]))
> }
> sapply(toyChecks, foo)
\end{Sinput}
\begin{Soutput}
 var1  var2  var3  var4  var5  var6 
 TRUE  TRUE  TRUE  TRUE  TRUE FALSE 
\end{Soutput}
\end{Schunk}
and we find that only a single variable in \R{toyData}, \R{var6} (for which all
observations have the value \R{"Irrelevant"}), is problem-free.




\subsection{Including cleanR documents in other files}
Sometimes, a \pkg{cleanR} document might be a useful addition to a
more general overview document, including also for example pairwise
association plots, time series plots, or exploratory analysis
results. To this end, it is possible to produce a \pkg{cleanR}
document that can readily be included in other \pkg{rmarkdown}
files. This is done by using the \R{standAlone} argument in \R{clean()},
which removes the preamble from the outputted \pkg{rmarkdown}
file. Please note, that it is still necessary to indicate which
\R{rmarkdown} type is being created; the pdf and html \pkg{rmarkdown}
styles are unfortunately not identical.

If it is important that the embedded \pkg{cleanR} document can be
rendered to either of these two file types, we recommend setting
\R{twoCols = FALSE} and \R{output = html} in \R{clean()}, thereby
essentially removing almost all output type specific formatting code
from the generated \pkg{rmarkdown} file.

On the other hand, if a pdf document is to be produced, a few extra
lines need to be added to the preamble of the master \pkg{rmarkdown}
document --- otherwise, the two-column layout code will produce an
error. The following is an example of how such a master document
preamble YAML might look like and how the \R{cleanR\_toyData.Rmd} file can
then be included:
{\small
\begin{Verbatim}
---
output: pdf_document
documentclass: report
header-includes:
  - \renewcommand{\chaptername}{Part}
  - \newcommand{\fullline}{\noindent\makebox[\linewidth]{\rule{\textwidth}{0.4pt}}}
  - \newcommand{\bminione}{\begin{minipage}{0.75 \textwidth}}
  - \newcommand{\bminitwo}{\begin{minipage}{0.25 \textwidth}}
  - \newcommand{\emini}{\end{minipage}}
---

```{r, child = 'cleanR_toyData.Rmd'}
```
\end{Verbatim}
}
%\hl{Use proper formatting here. How do we do non-R code?}

In the this example, the \R{cleanR\_toyData.Rmd} file could have been created as follows:
\begin{Schunk}
\begin{Sinput}
> clean(toyData, standAlone = FALSE)
\end{Sinput}
\end{Schunk}
and the more minimal, html-style \pkg{rmarkdown} file described above can be produced using
\begin{Schunk}
\begin{Sinput}
> clean(toyData, standAlone = FALSE, output = "html", twoCols = FALSE)
\end{Sinput}
\end{Schunk}
Note that with the latter option, no special YAML preamble is needed in the \R{rmarkdown} document.
% \hl{don't end on code.}



\section{Concluding remarks}
\label{conclusion}
%\hl{Is this a thing in this journal? Otherwise, we might want to make some final remarks in the previous sections. Feels awkward to end with a bunch of code and some super specific examples...}

In this paper we have introduced the \proglang{R} package \pkg{cleanR}
for performing reproducible error detection and data cleaning
summaries. The package provides a general and extendable framework for
identifying potential errors and for creating human-readable summary
documents that will help investigators to identify possible errors in
the data. 


While the the current release is stable, the authors have an interest
in further developing the functionality of \pkg{cleanR} by providing
more summary, visual, and check function as part of the default
package. We are also currently considering adding options to handle
repeated measurement, where the visualizations --- in particular ---
might be improved by visualizing measurements over time. In addition,
an online \pkg{shiny} application where users that are not
\proglang{R}-savvy can upload their data and get a data cleaning
document is currently planned.



\nocite{R}
\nocite{shiny}
\nocite{cleanR}
\nocite{rmarkdown}
\nocite{ggplot2}
\nocite{plyr}
\nocite{data.table}
\nocite{validate}
\nocite{editrules}
\nocite{janitor}
\nocite{DataCombine}
\nocite{txtplot}

% \bibliographystyle{jss}
\bibliography{foo}


\appendix
\newpage

%\includepdf[scale=0.8,clip,trim=0cm 0cm 0cm
%2cm,pages={1},pagecommand={\section{appendix}\subsection{par‌​t1}}]{pdfdocument} \includepdf[sc%ale=0.8,clip,trim=0cm
%0cm 0cm 2cm,pages={2-5},pagecommand={}]{pdfdocument}

% \section{Appendix A: cleaning the toyData data frame} \label{sec:appendix1}
\includepdf[pages=2, pagecommand={\section{Appendix A: cleaning the toyData data frame} \label{sec:appendix1}}, frame=true, noautoscale=true, scale=0.7]{cleanR_toyData.pdf}
\includepdf[pages=3-4, pagecommand={}, frame=true, noautoscale=true, scale=0.7,pagecommand={}]{cleanR_toyData.pdf}



\includepdf[pages=2, pagecommand={\section{Appendix B: User-extended cleaning of
  exampleData} \label{sec:appendix2}}, frame=true, noautoscale=true, scale=0.7]{cleanR_exampleData.pdf}
\includepdf[pages=3-, pagecommand={}, frame=true, noautoscale=true, scale=0.7]{cleanR_exampleData.pdf}
%\hl{Data cleaning with user supplied extensions here}

\end{document}